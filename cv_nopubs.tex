% Peter Williams' LaTeX CV template.
% This work is dedicated to the public domain.

\documentclass[letterpaper,11pt]{article}

% Basic settings for your convenience!

\newcommand{\myname}{Eric W. Koch}
\newcommand{\myaffil}{Dept. of Physics, University of Alberta}
\newcommand{\myemail}{ekoch@ualberta.ca}
\newcommand{\mypostlineone}{4-181 CCIS, University of Alberta}
\newcommand{\mypostlinetwo}{Edmonton, AB T6G 2E1}
\newcommand{\mysite}{e-koch.github.io}
\newcommand{\myorcid}{https://orcid.org/0000-0001-9605-780X}
% \newcommand{\mymobile}{+1 617 922 2689}

% Here begins a ton of boilerplate to set up the document to look nice.
% %%%%%%%%%%%%%%%%%%%%%%%%%%%%%%%%%%%%%%%%

% margins
\usepackage[margin=0.5in,bottom=0.75in]{geometry}

% nice links
\usepackage[colorlinks,urlcolor=blue]{hyperref}
\urlstyle{sf}

% needed for the fancy footer
\usepackage{fancyhdr,lastpage}

% I find that things look better this way.
\linespread{0.85}

% Layout. Very little comes in bare paragraphs. Most of the content is in
% lists, and we use enumitem to control the spacing.

\usepackage{enumitem}

\setlength{\parindent}{0pt}
\newlength{\mainindent} \setlength{\mainindent}{12pt}
\newlength{\contentindent} \setlength{\contentindent}{19ex}

% "nosep" is shorthand here, but may have compatibility issues.
\setlist[itemize]{topsep=0pt,partopsep=0pt,parsep=0pt,itemsep=0pt}

\newenvironment{datelist}{
  \begingroup
  \raggedright
  \begin{description}[labelindent=\mainindent,leftmargin=\contentindent,
      style=sameline,font=\normalfont,topsep=0pt,partopsep=0pt,parsep=0pt,
      itemsep=4pt]
}{
  \end{description}
  \endgroup
}

\newenvironment{publist}{
  \begingroup
  \raggedright
  \begin{description}[leftmargin=4ex,style=sameline]
}{
  \end{description}
  \endgroup
}

% Customize font / spacing of section titles
\usepackage[nobottomtitles*]{titlesec}
\renewcommand{\bottomtitlespace}{0.1\textheight}
\titleformat{\section}{\normalfont\large\bfseries}{\thesection}{1em}{}
\titlespacing*{\section}{0pt}{2.5ex plus 1ex minus 0.2ex}{1ex plus 0.2ex}

% %%%%%%%%%%%%%%%%%%%%%%%%%%%%%%%%%%%%%%%%
% That's the end of the boilerplate. Now come some things that don't involve
% any templating.

\begin{document}

% running footer
\pagestyle{fancy}
\lhead{} \chead{} \rhead{} \renewcommand{\headrule}{\relax}
\lfoot{\textsc{Curriculum Vit\ae}}
\cfoot{\thepage/\pageref*{LastPage}}
\rfoot{\textsc{\myname}}

% title
\begin{center}
\textbf{\Large \myname} \\
{\large Curriculum Vit\ae}
\end{center}

\medskip

% contact info, TeXified as a table
\begin{tabular*}{\textwidth}{@{\extracolsep{\fill}}lr}
\myaffil &
 \textsf{\href{mailto:\myemail}{\myemail}} \\
\mypostlineone &
 \url{\mysite} \\
\mypostlinetwo &
  \textsf{ORCID: \href{\myorcid}{0000-0001-9605-780X}} \\
 % Mobile: \mymobile
\end{tabular*}

\medskip

% %%%%%%%%%%%%%%%%%%%%%%%%%%%%%%%%%%%%%%%%
% Here's where we actually start using the templates.

Updated % note: must have this awkward newline for the simpleminded templater:
Oct 28, 2019.
% The latest version of this document is online at the website listed above.

\section*{Education}
\begin{datelist}
% This is entered manually to get the extra PhD info. It's not like this list
% is expected to change ...
\item[2016-expected July 2020]
  \emph{University of Alberta} \\
  PhD. (Physics) \\
  Thesis: {``The Molecular and Atomic Interstellar Medium in the Local Group''} \\
  Adviser: Prof. Erik Rosolowsky
\item[2014-2016]
  \emph{University of Alberta} \\
  MSc. (Physics) \\
  Thesis: {``The Atomic Interstellar Medium in M33''} \\
  Adviser: Prof. Erik Rosolowsky
\item[2010-2014]
  \emph{University of British Columbia} \\
  Hon. BSc. (Physics)
\end{datelist}



\section*{Employment}
\begin{datelist}
\item[2014\textendash{}present] \emph{University of Alberta} \\ Graduate Research and Teaching Assistant
\item[2013-2014] \emph{University of British Columbia, Okanagan} \\ Undergraduate Research Assistant with Prof. Jason Loeppky
\item[2012] \emph{University of British Columbia, Okanagan} \\ Undergraduate Work-Study Program with Prof. Erik Rosolowsky
\item[2011-2014] \emph{University of British Columbia, Okanagan} \\ Undergraduate Teaching Assistant
\end{datelist}

% \section*{Research Interests}
% \begin{itemize}
% \item Molecular cloud evolution, formation, and destruction
% \item Sources of turbulent driving, and the turbulent cascade from galaxy to cloud scales.
% \item Developing diagnostics of ISM structure
% \item
% \end{itemize}


\section*{Awards}
\begin{datelist}
\item[2019] \emph{University of Alberta} \\ Andrew Stewart Memorial Graduate Prize
\item[2019] \emph{University of Alberta/The Ohio State University} \\ Natural Sciences and Engineering Research Council of Canada Michael Smith Foreign Study Supplements with Prof. Adam Leroy
\item[2018] \emph{University of Alberta} \\ Queen Elizabeth II Graduate Scholarship - Doctorate
\item[2017-2019] \emph{University of Alberta} \\ Natural Sciences and Engineering Research Council of Canada Alexander Graham Bell Canada Graduate Scholarship - Doctorate
\item[2016] \emph{University of Alberta} \\ Natural Sciences and Engineering Research Council of Canada Postgraduate Scholarship - Doctorate
\item[2015] \emph{University of Alberta} \\ Queen Elizabeth II Graduate Scholarship - Masters
\item[2010-2014] \emph{University of British Columbia, Okanagan} \\ Deputy Vice Chancellor Scholarship
\item[2014] \emph{University of British Columbia, Okanagan} \\ Distinguished Graduate Award - Physics, Math, Statistics \& Computer Science
\item[2014] \emph{University of Alberta} \\ Natural Sciences and Engineering Research Council of Canada Alexander Graham Bell Canada Graduate Scholarship - Masters
\item[2013] \emph{University of Alberta} \\ Natural Sciences and Engineering Research Council of Canada Undergraduate Summer Research Award with Prof. Craig Heinke
\item[2013] \emph{University of British Columbia, Okanagan} \\ Top Oral Presenter - UBC-O Undergraduate Research Conference
\item[2013] \emph{University of British Columbia, Okanagan} \\ Upper Year Physics Award
\item[2012] \emph{University of British Columbia, Okanagan} \\ Natural Sciences and Engineering Research Council of Canada Undergraduate Summer Research Award with Prof. Erik Rosolowsky
\item[2010] \emph{University of British Columbia, Okanagan} \\ President's Entrance Scholarship
\end{datelist}

% FORMAT \item[|date|] \textbf{|am_pi|} - \emph{|what|} \\ |source| \\ \textbf{|amount|}

% \section*{External Funding}
% \begin{datelist}
% RMISCLIST grant
% \end{datelist}




\section*{Professional Talks}
\begin{datelist}
\item[2019 September] \emph{So-Star, Paris, France} \\ ``HI \& CO kinematics on molecular cloud scales in the Local Group''
\item[2019 April] \emph{Center for Astrophysics, Cambridge, USA} \\ ``Connecting atomic and molecular ISM kinematics on cloud scales in M33''
\item[2019 April] \emph{Green Bank Telescope, Green Bank, USA} \\ ``Connecting atomic and molecular ISM kinematics on cloud scales in M33''
\item[2019 March] \emph{NRAO, Charlottesville, USA} \\ ``Connecting atomic and molecular ISM kinematics on cloud scales in M33''
\item[2019 March] \emph{University of Texas, Austin, USA} \\ ``Connecting atomic and molecular ISM kinematics on cloud scales in M33''
\item[2019 January] \emph{Big Apple Magnetic Fields Workshop, New York, USA} \\ ``Turbustat: Python-based turbulence statistics''
\item[2018 August] \emph{CHANG-ES Team Meeting, Calgary, Canada} \\ ``De-obfuscating HI \& CO Comparisons in M33''
\item[2018 July] \emph{PHAT/M33 Team Meeting, Ringberg, Germany} \\ ``Atomic Gas in M31 and M33''
\item[2018 May] \emph{Olympian Symposium, Paralia Katerini, Greece} \\ ``Spatially-Varying Turbulent Properties in M33''
\item[2017 June] \emph{Canadian Astronomical Society (CASCA) Meeting, Edmonton, Canada} \\ ``Linking the Atomic and Molecular ISM in M33''
\item[2016 August] \emph{Lorentz Centre - Apples to Apples Workshop} \\ ``Identifying Tools for Comparing Simulations and Observations of Star-forming Regions''
\item[2016 February] \emph{Max Planck Institute for Extraterrestrial Physics} \\ ``Comparing Simulations and Observations of Star Formation using Experimental Design''
\item[2016 February] \emph{Max Planck Institute for Radio Astronomy} \\ ``Comparing Simulations and Observations of Star Formation using Experimental Design''
\item[2015 May] \emph{Florence Simulations-Observations Workshop (Florence, Italy)} \\ ``Critically Comparing Simulations and Observations of Star Formation''
\item[2014 April] \emph{UBC-O Undergraduate Research Conference (Kelowna, Canada)} \\ ``Filament Identification through Mathematical Morphology''
\item[2013 November] \emph{UBC-O Brown Bag Series (Kelowna, Canada)} \\ ``A New Low-Mass X-Ray Binary in the Galactic Centre''
\item[2013 April] \emph{UBC-O Undergraduate Research Conference (Kelowna, Canada)} \\ ``Converging Flows in Star-Forming Regions''
\end{datelist}

\section*{Poster Presentations}
\begin{datelist}
\item[2019 June] \emph{Linking the Milky Way and Nearby Galaxies, Helsinki, Finland} \\ ``Connecting atomic and molecular ISM kinematics on cloud scales in M33''
\item[2013 June] \emph{Canadian Astronomical Society Conference (Vancouver, Canada)} \\ ``Converging Flows in Star-Forming Regions''
\end{datelist}

% \section*{Professional Talks --- Invited}
% \begin{datelist}
% RMISCLIST_IF talk invited
% \end{datelist}

% \section*{Professional Talks --- Other}
% \begin{datelist}
% RMISCLIST_IF_NOT talk invited
% \end{datelist}

\section*{Observing Experience}
\begin{itemize}
\item Three VLA projects as PI (180 hours awarded; 16B-236, 16B-242, 17B-162); one as co-I (24 hours awarded; 19B-037)
\item One ALMA project as PI (8 hours awarded; 2019.1.01039.S); two as co-I (22 hours awarded; 2017.1.00901.S, 2019.1.01182.S)
\item One GBT project as PI (41 hours awarded; 19B-221)
\item One NOEMA project as co-I (16 hours awarded; W15BR)
\end{itemize}

\section*{Professional Service}
\begin{datelist}

\item[2018-Present] Referee for Monthly Notices of the Royal Astronomical Society
\item[2017-Present] Seminar and journal club organizer for UAlberta Astronomy Group
\item[2017-Present] Student Member of Canadian Astronomical Society (CASCA)
\item[2016-2017] UAlberta representative on the Canadian Astronomical Society Graduate Student Committee
\item[2013-2014] UBC-Okanagan Physics representative on Quantitative Sciences Course Union Council

\end{datelist}


\section*{Research Advising}

I have acted as a research advisor for three undergraduate students at the University of Alberta, supervised by Prof. Erik Rosolowsky.
\vspace{0.1in}
\begin{datelist}
% This is entered manually to get the extra PhD info. It's not like this list
% is expected to change ...
\item[Summer 2018]
  \emph{Interpreting filaments in three dimensions} \\
  Dewanshu Haswani \\
  MITACS Internship
\item[Fall 2018]
  \emph{Spiral Arm Propagation in M33 and its Implications on Molecular Cloud Formation} \\
  Steffen Senchyna \\
  Physics 499 Honours Research Project
\item[Fall 2018]
  \emph{ISM Properties near Supernova Remnants in M33} \\
  Weizhuo Zhang \\
  Physics 499 Honours Research Project
\end{datelist}


% FORMAT \item[|date|] \emph{|where|} \\ |what|

% \section*{Teaching}
% \begin{datelist}
% RMISCLIST teaching
% \end{datelist}

% FORMAT \item[|date|] |what|

\section*{Outreach}
\begin{datelist}
\item[2019 October] \emph{Edmonton, Canada} \\ ``Judge for NASA/CSA Space Apps Challenge''
\item[2019 May] \emph{Pint of Science, Edmonton, Canada} \\ ``Frigid Fuel for Star Formation''
\item[2018 December] \emph{Royal Astronomical Society of Canada (Edmonton Centre)} \\ ``Unravelling Star Formation''
\item[2018 May] \emph{Northern Alberta Radio Club (Edmonton)} \\ ``Viewing the Sky with Radio Interferometry''
\item[2017 February] \emph{University of Alberta Observatory Public Observing Night} \\ ``Blowing Bubbles in a Galaxy''
\item[2016 - 2018; 2019-Present] \emph{University of Alberta Observatory} \\ ``Public Observing \& School Tours''
\end{datelist}

\end{document}
