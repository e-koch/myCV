% Peter Williams' LaTeX CV template.
% This work is dedicated to the public domain.

\documentclass[letterpaper,11pt]{article}

% Basic settings for your convenience!

\newcommand{\myname}{Eric W. Koch}
\newcommand{\myaffil}{Dept. of Physics, University of Alberta}
\newcommand{\myemail}{ekoch@ualberta.ca}
\newcommand{\mypostlineone}{4-181 CCIS, University of Alberta}
\newcommand{\mypostlinetwo}{Edmonton, AB T6G 2E1}
\newcommand{\mysite}{e-koch.github.io}
% \newcommand{\mymobile}{+1 617 922 2689}

% Here begins a ton of boilerplate to set up the document to look nice.
% %%%%%%%%%%%%%%%%%%%%%%%%%%%%%%%%%%%%%%%%

% margins
\usepackage[margin=0.5in,bottom=0.75in]{geometry}

% nice links
\usepackage[colorlinks,urlcolor=blue]{hyperref}
\urlstyle{sf}

% needed for the fancy footer
\usepackage{fancyhdr,lastpage}

% I find that things look better this way.
\linespread{0.85}

% Layout. Very little comes in bare paragraphs. Most of the content is in
% lists, and we use enumitem to control the spacing.

\usepackage{enumitem}

\setlength{\parindent}{0pt}
\newlength{\mainindent} \setlength{\mainindent}{12pt}
\newlength{\contentindent} \setlength{\contentindent}{19ex}

% "nosep" is shorthand here, but may have compatibility issues.
\setlist[itemize]{topsep=0pt,partopsep=0pt,parsep=0pt,itemsep=0pt}

\newenvironment{datelist}{
  \begingroup
  \raggedright
  \begin{description}[labelindent=\mainindent,leftmargin=\contentindent,
      style=sameline,font=\normalfont,topsep=0pt,partopsep=0pt,parsep=0pt,
      itemsep=4pt]
}{
  \end{description}
  \endgroup
}

\newenvironment{publist}{
  \begingroup
  \raggedright
  \begin{description}[leftmargin=4ex,style=sameline]
}{
  \end{description}
  \endgroup
}

% Customize font / spacing of section titles
\usepackage[nobottomtitles*]{titlesec}
\renewcommand{\bottomtitlespace}{0.1\textheight}
\titleformat{\section}{\normalfont\large\bfseries}{\thesection}{1em}{}
\titlespacing*{\section}{0pt}{2.5ex plus 1ex minus 0.2ex}{1ex plus 0.2ex}

% %%%%%%%%%%%%%%%%%%%%%%%%%%%%%%%%%%%%%%%%
% That's the end of the boilerplate. Now come some things that don't involve
% any templating.

\begin{document}

% running footer
\pagestyle{fancy}
\lhead{} \chead{} \rhead{} \renewcommand{\headrule}{\relax}
\lfoot{\textsc{Curriculum Vit\ae}}
\cfoot{\thepage/\pageref*{LastPage}}
\rfoot{\textsc{\myname}}

% title
\begin{center}
\textbf{\Large \myname} \\
{\large Curriculum Vit\ae}
\end{center}

\medskip

% contact info, TeXified as a table
\begin{tabular*}{\textwidth}{@{\extracolsep{\fill}}lr}
\myaffil &
 \textsf{\href{mailto:\myemail}{\myemail}} \\
\mypostlineone &
 \url{\mysite} \\
% \mypostlinetwo &
 % Mobile: \mymobile
\end{tabular*}

\medskip

% %%%%%%%%%%%%%%%%%%%%%%%%%%%%%%%%%%%%%%%%
% Here's where we actually start using the templates.

Updated % note: must have this awkward newline for the simpleminded templater:
TODAY.
% The latest version of this document is online at the website listed above.

\section*{Education}
\begin{datelist}
% This is entered manually to get the extra PhD info. It's not like this list
% is expected to change ...
\item[2014-2016]
  \emph{University of Alberta} \\
  MSc. (Physics) \\
  Thesis: {``Triggered Star Formation in M33''} \\
  Adviser: Erik Rosolowsky
\item[2010-2014]
  \emph{University of British Columbia} \\
  Hon. BSc. (Physics)
\end{datelist}

FORMAT \item[|date|] \emph{|where|} \\ |what|

\section*{Employment}
\begin{datelist}
RMISCLIST job
\end{datelist}

\section*{Fellowships and Awards}
\begin{datelist}
RMISCLIST award
\end{datelist}

FORMAT \item[|date|] \textbf{|am_pi|} - \emph{|what|} \\ |source| \\ \textbf{|amount|}

\section*{External Funding}
\begin{datelist}
RMISCLIST grant
\end{datelist}

FORMAT \item[|rev_number|.] |full_authors|. \textit{``|quotable_title|.''} |year|, |lcite|.

\section*{Submitted Publications}

\begin{publist}
PUBLIST refpreprint_rev
\end{publist}

\section*{Refereed Publications}

\begin{publist}
PUBLIST refereed_rev
\end{publist}

% \section*{Research Interests}
% \begin{itemize}
% \item Magnetic activity of low-mass stars and brown dwarfs
% \item The dynamic radio sky: overall properties, known and potential sources,
%     surveys
% \item Radio interferometric software, analysis and techniques
% \end{itemize}

% \section*{Observing Experience}
% \begin{itemize}
% \item Centimeter radio interferometry (Allen Telescope Array, VLA)
% \item X-ray and UV imaging (\textit{Chandra}, \textit{Swift})
% \item Optical imaging and spectroscopy (CTIO/Blanco~4m, Lick/Shane~3m,
%     Lick/Nickel~1m, Lick/CAT~1m)
% \end{itemize}

FORMAT \item[|date|] \emph{|where|} \\ ``|what|''

\section*{Professional Talks}
\begin{datelist}
RMISCLIST talk
\end{datelist}

\section*{Poster Presentations}
\begin{datelist}
RMISCLIST poster
\end{datelist}

% \section*{Professional Talks --- Invited}
% \begin{datelist}
% RMISCLIST_IF talk invited
% \end{datelist}

% \section*{Professional Talks --- Other}
% \begin{datelist}
% RMISCLIST_IF_NOT talk invited
% \end{datelist}

% FORMAT \item[|date|] \emph{|where|} \\ |what|

% \section*{Teaching}
% \begin{datelist}
% RMISCLIST teaching
% \end{datelist}

% FORMAT \item[|date|] |what|

% \section*{Outreach}
% \begin{datelist}
% RMISCLIST outreach
% \end{datelist}

\end{document}
